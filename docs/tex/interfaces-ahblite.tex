\subsection{AHB-Lite Interface}

The AHB-Lite interface is a regular AHB-Lite slave port. All signals are
supported. See the
\emph{\href{https://www.arm.com/products/system-ip/amba-specifications}{AMBA
3 AHB-Lite Specification}} for a complete description of the signals.

\begin{longtable}[c]{@{\extracolsep{\fill}}cccl@{}}	
		\toprule 
		\textbf{Port} & \textbf{Size} & \textbf{Direction} & \textbf{Description}\\
		\midrule
		\endhead 
		\texttt{HRESETn} & 1 & Input & Asynchronous active low reset\\
		\texttt{HCLK} & 1 & Input & Clock Input\\
		\texttt{HSEL} & 1 & Input & Bus Select\\
		\texttt{HTRANS} & 2 & Input & Transfer Type\\
		\texttt{HADDR} & \texttt{HADDR\_SIZE} & Input & Address Bus\\
		\texttt{HWDATA} & \texttt{HDATA\_SIZE} & Input & Write Data Bus\\
		\texttt{HRDATA} & \texttt{HDATA\_SIZE} & Output & Read Data Bus\\
		\texttt{HWRITE} & 1 & Input & Write Select\\
		\texttt{HSIZE} & 3 & Input & Transfer Size\\
		\texttt{HBURST} & 3 & Input & Transfer Burst Size\\
		\texttt{HPROT} & 4 & Input & Transfer Protection Level\\
		\texttt{HREADYOUT} & 1 & Output & Transfer Ready Output\\
		\texttt{HREADY} & 1 & Input & Transfer Ready Input\\
		\texttt{HRESP} & 1 & Output & Transfer Response\\
		\bottomrule 	
	\caption{PLIC Interface Signals}
	\label{tab:AHBIF}
\end{longtable}

\subsubsection{HRESETn}

When the active low asynchronous \texttt{HRESETn} input is asserted
(`0'), the interface is put into its initial reset state.

\subsubsection{HCLK}

\texttt{HCLK} is the interface system clock. All internal logic for the
AHB-Lite interface operates at the rising edge of this system clock and
AHB bus timings are related to the rising edge of \texttt{HCLK}.

\subsubsection{HSEL}

The AHB-Lite interface only responds to other signals on its bus -- with
the exception of the global asynchronous reset signal \texttt{HRESETn}
-- when \texttt{HSEL} is asserted (`1'). When \texttt{HSEL} is negated
(`0') the interface considers the bus \texttt{IDLE}.

\subsubsection{HTRANS}

<<<<<<< HEAD
HTRANS indicates the type of the current transfer as shown in Table \ref{tab:HTRANS}
\being{comment}
This shows as [tab:HTRANS]: on screen (DATASHEET.md)
\end{comment}

=======
HTRANS indicates the type of the current transfer as shown in 
\ifdefined\MARKDOWN
Table \ref{tab:HTRANS}
\else
the table below:
\fi
>>>>>>> 09c42428a2dc64a54dab1ef71b84f36f822d116e

\begin{longtable}[c]{@{\extracolsep{\fill}}ccp{7cm}}	
		\toprule 
		\textbf{HTRANS} & \textbf{Type} & \textbf{Description}\\
		\midrule
		\endhead 
		00 & \texttt{IDLE} & No transfer required\\
		01 & \texttt{BUSY} & Connected master is not ready to accept data, but intents to continue the current burst.\\
		10 & \texttt{NONSEQ} & First transfer of a burst or a single transfer\\
		11 & \texttt{SEQ} & Remaining transfers of a burst\\
		\bottomrule 	
	\caption{HTRANS Signal Types}
	\label{tab:HTRANS}
\end{longtable}

\subsubsection{HADDR}

\texttt{HADDR} is the address bus. Its size is determined by the
\texttt{HADDR\_SIZE} parameter and is driven to the connected
peripheral.

\subsubsection{HWDATA}

\texttt{HWDATA} is the write data bus. Its size is determined by the
\texttt{HDATA\_SIZE} parameter and is driven to the connected
peripheral.

\subsubsection{HRDATA}

\texttt{HRDATA} is the read data bus. Its size is determined by the
\texttt{HDATA\_SIZE} parameter and is sourced by the connected
peripheral.

\subsubsection{HWRITE}

\texttt{HWRITE} is the read/write signal. \texttt{HWRITE} asserted (`1')
indicates a write transfer.

\subsubsection{HSIZE}

<<<<<<< HEAD
\texttt{HSIZE} indicates the size of the current transfer as shown in table \ref{tab:HSIZE}:

\being{comment}
This shows as [tab:HSIZE]: on screen (DATASHEET.md)
\end{comment}
=======
\texttt{HSIZE} indicates the size of the current transfer as shown in
\ifdefined\MARKDOWN
Table \ref{tab:HSIZE}
\else
the table below:
\fi
>>>>>>> 09c42428a2dc64a54dab1ef71b84f36f822d116e

\begin{longtable}[c]{@{\extracolsep{\fill}}ccl}	
		\toprule 
		\textbf{HSIZE} & \textbf{Size} & \textbf{Description}\\
		\midrule
		\endhead 
		000 & 8 bit & Byte\\
		001 & 16 bit & Half Word\\
		010 & 32 bit & Word\\
		011 & 64 bits & Double Word\\
		100 & 128 bit &\\
		101 & 256 bit &\\
		110 & 512 bit &\\
		111 & 1024 bit &\\
		\bottomrule 	
	\caption{HSIZE Values}
	\label{tab:HSIZE}
\end{longtable}

\subsubsection{HBURST}

HBURST indicates the transaction burst type -- a single transfer or part
of a burst.

\begin{longtable}[c]{@{\extracolsep{\fill}}ccl}	
		\toprule 
		\textbf{HBURST} & \textbf{Type} & \textbf{Description}\\
		\midrule
		\endhead 
		000 & \texttt{SINGLE} & Single access**\\
		001 & \texttt{INCR} & Continuous incremental burst\\
		010 & \texttt{WRAP4} & 4-beat wrapping burst\\
		011 & \texttt{INCR4} & 4-beat incrementing burst\\
		100 & \texttt{WRAP8} & 8-beat wrapping burst\\
		101 & \texttt{INCR8} & 8-beat incrementing burst\\
		110 & \texttt{WRAP16} & 16-beat wrapping burst\\
		111 & \texttt{INCR16} & 16-beat incrementing burst\\
		\bottomrule 	
	\caption{HBURST Types}
	\label{tab:HBURST}
\end{longtable}

\subsubsection{HPROT}

The \texttt{HPROT} signals provide additional information about the bus
transfer and are intended to implement a level of protection.

\begin{longtable}[c]{@{}lccl}	
		\toprule 
		& \textbf{Bit\#} & \textbf{Value} & \textbf{Description}\\
		\midrule
		\endhead 
		& 3 & 1 & Cacheable region addressed\\
		& & 0 & Non-cacheable region addressed\\
		& 2 & 1 & Bufferable\\
		& 0 & Non-bufferable\\
		& 1 & 1 & Privileged Access\\
		& & 0 & User Access\\
		& 0 & 1 & Data Access\\
		& & 0 & Opcode fetch\\
		\bottomrule 	
	\caption{HPROT Indicators}
	\label{tab:HPROT}
\end{longtable}


\subsubsection{HREADYOUT}

\texttt{HREADYOUT} indicates that the current transfer has finished.
Note, for the AHB-Lite PLIC this signal is constantly asserted as the
core is always ready for data access.

\subsubsection{HREADY}

\texttt{HREADY} indicates whether or not the addressed peripheral is
ready to transfer data. When \texttt{HREADY} is negated (`0') the
peripheral is not ready, forcing wait states. When \texttt{HREADY} is
asserted (`1') the peripheral is ready and the transfer completed.

\subsubsection{HRESP}

\texttt{HRESP} is the instruction transfer response and indicates OKAY
(`0') or ERROR (`1').

\newpage
